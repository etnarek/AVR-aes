\documentclass[a4paper,10pt]{article}
\usepackage[french]{babel}
\usepackage[utf8]{inputenc}
\usepackage[left=2.5cm,top=2cm,right=2.5cm,nohead]{geometry}
\usepackage{url}
\usepackage[T1]{fontenc}
\usepackage{float}
\usepackage{afterpage}
\usepackage{amsmath}
\usepackage{graphicx}
\usepackage{tabularx}
\usepackage{csquotes}
\usepackage{fullpage}
\usepackage{blindtext}
\usepackage[section]{placeins}

\author{Romain \textsc{Fontaine}}
\title{Implémentation de Schedaes}

\begin{document}
\maketitle

\section{Introduction:}
c'est quoi aes, scedaes
c'est quoi le side-channel, pourquoi c'est important

\section{Présentation de la solution}
comment t'as bossé (c -> asm, tests en pytons, bla blabla)
tu parles de ton random et du seed (qui est non sécure)

\section{conclusion}
ce que ça a donné, ce que tu ferais si tu avais plus de temps
ça pourrait être pas mal d'implémenter le random comme ça
\end{document}
