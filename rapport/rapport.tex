\documentclass[letterpaper]{article}
\usepackage{natbib,alifexi}
\usepackage[french]{babel}
\usepackage[utf8]{inputenc}
\usepackage{url}
\usepackage[T1]{fontenc}
\usepackage{amsmath}
\usepackage{graphicx}
\usepackage{tabularx}
\usepackage{csquotes}

\author{Romain \textsc{Fontaine}$^{1}$\\
    Superviseurs : Olivier \textsc{Markowitch}, Stephane Fernandes \textsc{Medeiros}\\
    \mbox{}\\
    $^1$Université Libre de Bruxelles, Département d’Informatique \\
    rfontain@ulb.ac.be
}
\title{Implémentation en \texttt{AVR} d'un algorithme de chiffrement :\\ \textsc{Schedaes}}
\date{}

\begin{document}
\maketitle

\section{Introduction:}
c'est quoi aes, scedaes
c'est quoi le side-channel, pourquoi c'est important

\section{Présentation de la solution}
comment t'as bossé (c -> asm, tests en pytons, bla blabla)
tu parles de ton random et du seed (qui est non sécure)

\section{conclusion}
ce que ça a donné, ce que tu ferais si tu avais plus de temps
ça pourrait être pas mal d'implémenter le random comme ça
\end{document}
