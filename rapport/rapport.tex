\documentclass[letterpaper]{article}
\usepackage{natbib,alifexi}
\usepackage[french]{babel}
\usepackage[utf8]{inputenc}
\usepackage{url}
\usepackage[T1]{fontenc}
\usepackage{amsmath}
\usepackage{graphicx}
\usepackage{tabularx}
\usepackage{csquotes}
\usepackage[]{algorithmic}
\usepackage[]{algorithm}
\usepackage{hyperref}

\author{Romain \textsc{Fontaine}$^{1}$\\
    Superviseurs : Olivier \textsc{Markowitch}, Stephane Fernandes \textsc{Medeiros}\\
    \mbox{}\\
    $^1$Université Libre de Bruxelles, Département d’Informatique \\
    rfontain@ulb.ac.be
}
\title{Implémentation en assembleur d'un algorithme de chiffrement :\\ SchedAES}
\date{}

\floatname{algorithm}{Algorithme}
\setlength{\parskip}{0.5em}

\begin{document}
\maketitle

\section{Introduction:}
La sécurité est une section importante de la Science informatique. De nombreux systèmes se font compromettre alors qu'ils contiennent souvent des informations sensibles. Pour éviter de voir ces secrets divulgués publiquement, il faut faire en sorte que des attaquants ne puissent accéder à ces informations.

Ce papier parlera de l'implémentation d'AES et de SchedAES pour un système embarqué (atmega328p ou arduino). L'implémentation finale se fera en assembleur pour micro contrôler AVR.

AES est un ``block-cipher'' ayant, dans notre cas, une clé et un bloc de 128 bits.
La CNSS (Committee on National Security Systems) le considère comme suffisamment sécurisé pour chiffrer des information classifiée ``secret'' dans sa version 128 bits (\cite{policy2003no}).
AES est un algorithme de chiffrement symétrique et donc la clé de chiffrement/déchiffrement doit rester bien secrète.

Cependant, ce n'est pas suffisant dans notre cas puisque qu'un attaquant potentiel peut ouvrir le dispositif et faire une attaque par canaux auxiliaires sur celui-ci afin d'en récupérer la clé de chiffrement. Ce contre quoi AES n'est pas suffisamment robuste (\cite{Renauld2009}).

Une attaque par canaux auxiliaires est une attaque où l'attaquant profite du fait qu'il aie accès au dispositif de chiffrement pour faire de l'analyse afin d'en récupérer les clés de chiffrement/déchiffrement.
Pour se faire, il peut analyser le temps nécessaire à l'exécution de certaines opérations, analyse des émanations électromagnétique, analyse de la consommation,\ldots(\cite{zhou2005side})

Ce type d'attaque peut être appliquée sur des systèmes critique comme des Smart-Cards (\cite{chari1999cautionary}).

Il existe plusieurs manières de protéger un algorithme face à ces attaques:
\begin{itemize}
    \item[Ajouter du bruit :] faire des opération en parallèle de sorte à modifier la consommation ou les émanations électromagnétique par exemple.
    \item[Masquer :] Ajouter un masque à la clé et au texte à chiffrer afin de rendre plus difficile la récupération de la clé originelle.
    \item[Contrôler le temps d'exécution des opérations :] Changer la fréquence du processeur en cours d'exécution, sauter quelque cycle entre deux instructions ou changer l'ordre des instructions.
\end{itemize}

Il existe plusieurs variantes d'AES visant à le rendre plus robuste face au attaques par canaux auxiliaires.
Celle que nous allons voir ici est SchedAES.
Celle-ci rend l'ordre dans lequel les différentes opérations d'AES sont exécutées aléatoire.
Elle s'apparente au contremesures visant à contrôler le temps d'exécution des opérations.

\section{Description des algorithmes:}
\subsection{AES (ADVANCED ENCRYPTION STANDARD)}
Considérons le texte à chiffrer comme une matrice de 16 octets (4 lignes et 4 colonne).
AES se compose de quatre opération importante:
\begin{itemize}
    \item[SubBytes :] substitution de chaque octet du texte à chiffrer grâce à une matrice de substitution.
    \item[ShiftRows :] Décalage de chaque ligne de la matrice à chiffrer de $i$ colonnes ($i$ étant le numéro de la ligne).
    \item[MixColumns :] Multiplication des éléments de chaque colonne entre eux\footnote{Cette opération est plus complexe qu'une simple multiplication, il est conseillé de lire \cite{fips197} pour mieux l'appréhender.}.
    \item[AddRoundKey :] Ou exclusif entre la clé de chiffrement et le texte pour chacune des case de la matrice. La clé étant modifié à chaque tour de l'algorithme.
\end{itemize}

L'ordre de ces opérations est donné dans l'algorithme~\ref{alg:aes}.
\begin{algorithm}
    \caption{AES}
    \label{alg:aes}
    \begin{algorithmic}[1]
        \REQUIRE texte à chiffrer, clé de chiffrement
        \STATE AddRoundKey
        \STATE GenerateNextRoundKey
        \FOR{$i=0 \rightarrow 9$}
            \STATE SubBytes
            \STATE ShiftRows
            \STATE MixColumns
            \STATE AddRoundKey
            \STATE GenerateNextRoundKey
        \ENDFOR
        \STATE SubBytes
        \STATE ShiftRows
        \STATE AddRoundKey
    \end{algorithmic}
\end{algorithm}

GenerateNextRoundKey étant le calcul de la clé pour le prochain AddRoundKey en fonction de la clé courante.
Cette opération peut être faite qu'une seule fois au début en mettant les clé de tous les prochains tour en mémoire ou être fait à chaque tour de boucle.

AES est définis dans\cite{fips197}

\subsection{SchedAES}
SchedAES utilise les même opérations que AES. Cependant, comme l'ordre des opérations est aléatoire, il garde en mémoire un set $\Theta$ des prochaines opérations qu'il peut exécuter.

Lors de l'exécution de SchedAES, la prochaine opération à être appliquée à la matrice à chiffrer sera récupéré aléatoirement depuis ce set.
Une fois l'opération terminée, $\Theta$ sera mis à jour en fonction de ce qui a déjà été réalisé.
Ceci est fait tant que $\Theta$ n'est pas vide.
Il sera vide une fois que toute les opérations nécessaires pour chiffrer le message seront finies.

Les opérations ici ne seront plus considérés sur la matrice entière mais sur chaque case, ligne ou colonne séparément en fonction de l'opération.

On ajoute une opération à $\Theta$ lorsque certaines pré conditions sont validées.
$\Theta$ initialement rempli avec l'opération AddRoundKey pour chacune des cases.

Les pré conditions sont les suivantes:
\begin{itemize}
    \item[SubByte :] il est ajouté lorsque le AddRoundKey précédent sur la même case a été exécuté.
    \item[ShiftRow :] il n'est techniquement pas nécessaire de réellement exécuter cette opération. Ici, ShiftRow est ajouté à $\Theta$ lorsque touts les AddRoundKey sur la ligne correspondante ont été exécuté et que le ShiftRow précédent sur cette même ligne a aussi été exécuté.
    \item[MixColumn :] il est ajouté lorsque tous les SubByte appartenant à la colonne colonne impliquée ont été exécuté. Comme il se peut que tout les ShiftRows n'ont pas été exécutés, la colonne ou il faut vérifier le SubByte n'est pas toujours celle sur laquelle le MixColumn va s'appliquer. Il faut calculer le décalage.
    \item[AddRoundKey :] ils sont ajoutés par 4 pou une colonne lorsque qu'un MixColumn a été exécuté.Il est aussi ajouté après un SubByte lorsque qu'on est au 10\up{ème} tour pour celui-ci.
\end{itemize}

Pour savoir l'état d'avancement, il faudra conserver en mémoire une matrice d'état pour chacune de ces opérations pour savoir si elles ont été exécutées.

L'algorithme est expliqué plus en détail dans \cite{FernandesMedeiros2012}.

\section{Présentation de la solution}
Le développement pour système embarqué n'est pas toujours évident car il n'est pas toujours possible de déboguer lorsque qu'une erreur se produit.
C'est pourquoi nous avons décidés de travailler de la manière suivante.

Pour commencer, l'implémentation de AES en C compilé et exécuté sur ordinateur.
Lors de cette étape, AES fut découpé en petite composante facilement implémentable.
Du test unitaire fut appliqué sur chacune de ces fonction pour être certain qu'elles renvoyaient bien le bon résultat.
Une fois l'implémentation de l'algorithme terminé, il fut aisé de voir si tout fonctionnait bien grâce à ces test unitaires et si le résultat final pour un clair et une clé donné était correct.
Cette implémentation a été faite en C pour être le plus proche possible de l'assembleur.
Dans cette optique d'être le plus proche de l'assembleur, un effort a été fait pour avoir des structures de données les plus simples possible et d'avoir des instructions facilement transformable en assembleur plutôt que des instructions complexes non traduisibles.

Une fois ceci fait, le code a été modifié pour être utilisable sur arduino.
Pour ce faire, il a fallut remplacer tout les \texttt{printf} par des messages sur le port série.
C'est ainsi que les fonctions \texttt{USART\_send} et \texttt{USART\_receive} ont été implémentées du coté arduino pour faciliter l'utilisation de USART (Universal Synchronous Asynchronous Receiver Transmitter).
La librairie \texttt{Serial} n'étant pas accessible en assembleur.

Maintenant que l'arduino est capable de lire et d'écrire sur le port série, il a fallut faire en sorte que l'ordinateur puisse communiquer avec. Dans cette optique, les fichier \texttt{test.py} et \texttt{communique.py} ont été développé. Ceux-ci, avec l'aide de la libraire \texttt{pyserial} ont permis d'envoyer sur le port série les messages à chiffrer et de les récupérer une fois le traitement terminé. \texttt{communique.py} lisant sur \texttt{stdin} les messages à chiffrer avant de
les envoyer alors que \texttt{test.py} envoyait celui par défaut afin de vérifier que l'algorithme fonctionnait bien en sachant ce que celui-ci devait retourner et en le comparant avec ce qui fut renvoyé.\footnote{Il est à noter qu'un arduino peut mettre du temps à démarrer. Il a donc été décidé de, au lieu d'attendre un temps arbitrairement long et de soit patenter pour rien, soit d'envoyer le message à chiffrer trop tôt, d'envoyer un simple caractère (* en l'occurrence) à partir du
moment ou l'arduino est prêt. Les deux scripts python attendent donc ce caractère avant de commencer.}

Un \texttt{Makefile} fut aussi créé pour automatiser la compilation, le téléversement et le lancement du test. Une fois ceci terminé, il est devenu beaucoup plus aisé de modifier l'implémentation tout en vérifiant qu'elle reste fonctionnelle. 

Maintenant que tout ceci a été fait, il est possible de transformer le code C en assembleur.
Ceci à été réalisée en transformant toutes les fonction une par une.
Elles ont été écrites dans des fichiers \texttt{.s} afin de pouvoir continuer d'utiliser l'éditeur de lien de \texttt{gcc} ce qui permet d'avoir les fonctions d'initialisations propres aux plateforme utilisée générée et donc d'avoir un code \texttt{AVR} portable sur d'autre microprocesseurs également.
Les tests étant lancés à chaque nouvelle fonction transformée.

Les outils utilises pour compiler et mettre à jour le code sur l'arduino sont respectivement \texttt{avr-gcc} et \texttt{avrdude}.

L'implémentation de SchedAES fut fort similaire à l'exception qu'il a fallut ajouter de l'aléatoire.
Le générateur d'aléatoire utilisé est le ``Générateur congruentiel linéaire'' dont la formule est :
$$ X_{n+1} = (a \cdot X_n  + c) \text{ mod } m$$
Avec comme valeur: $a=45$, $c=77$ et $m=2^8$. Le m a été choisi si petit pour rentrer sur un octet et pour pouvoir facilement calculer le modulo vu qu'il suffit de faire un ``et logique'' avec 255.
Une plus grande analyse sur la génération d'aléatoire sera faite dans la section qui suit.

Comme nous pouvons le voir, le générateur d'aléatoire se base sur un nombre généré précédemment pour en créer un nouveau.
Il faut donc un ``seed'' (une valeur de départ) pour que celui-ci fonctionne.
Au début du développement, le seed a été mis à une valeur arbitraire en attendant de le générer d'une manière aléatoire.
Par la suite, les entrées analogiques de l'arduino ont été utilisée pour créer un seed aléatoirement.
En effet, lorsqu'une entrée analogique est dite ``flottante'' (n'est pas relié à un circuit), elle renvoie des valeurs aléatoires en fonction des champs électromagnétique dans les environs.

L'implémentation des deux algorithme est disponible sur github: \url{https://github.com/etnarek/AVR-aes}

\section{Discussion sur l'aléatoire}
\label{sec:rand}
Le générateur d'aléatoire utilisé est le Générateur congruentiel linéaire car il est suffisamment rapide et pas trop complexe à implémenter.
Cependant, ce n'est pas le meilleur car il n'est pas suffisamment sécurisé pour gérer de la cryptographie (\cite{stern1987secret} et \cite{Plumstead1983}).
Il aurait été meilleur d'utiliser des algorithmes comme l'algorithme de \cite{blum1984generate} mais qui est beaucoup plus lent à cause de l'exponentielle.
De plus, il serait intéressant de vérifier que ces algorithme soient bien résistant aux attaques par canaux auxiliaires.
Lire une valeur aléatoirement sur une entrée analogique n'est conseillé car il est possible de faire en sorte que celle-ci renvoie toujours la même valeur en la reliant à la terre par exemple.

Le seed est aussi à prendre en compte lors de la génération d'aléatoire.
En effet, si on fixais le seed à une valeur arbitraire dans le code pour l'arduino, celui-ci retournera à cette même valeur à chaque redémarrage de l'arduino.
Cela fait qu'à chaque fois, les instructions seront exécutées dans le même ordre.
La technique de lire sur une entrée analogique le seed (comme utilisé ici) n'est pas conseillé pour les même raisons que précédemment, il est aisé de choisir la valeur voulue.
La dernière possibilité évoquée dans cet article est de choisir aléatoirement une première valeur pour le seed et de la mettre dans l'EEPROM (Electrically Erasable Programmable Read-Only Memory) de l'arduino. Par la suite, à chaque génération d'un nouveau nombre, on modifie l'EEPROM de l'arduino avec celui-ci comme ça il est persistant entre chaque redémarrage.
Il y a juste lorsqu'on met la valeur pour la première fois dans l'arduino qu'il faut faire attention.
Cependant, nous considérons que l'attaquant n'y a pas accès à ce moment sinon il aurait accès à la clé de chiffrement par la même occasion.

\section{Conclusion}
ce que ça a donné, ce que tu ferais si tu avais plus de temps
ça pourrait être pas mal d'implémenter le random comme ça


\footnotesize
\bibliographystyle{apalike}
\bibliography{biblio}
\end{document}
